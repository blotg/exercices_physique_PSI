%%difficulte:   3
%%omis:         False
%%titre:        Chute d'un crayon
%%classes:      [PCSI]

\exercice{Chute d'un crayon}

Un crayon à papier est modélisé par une un segment de droite sur lequel la masse du crayon est répartie uniformément.

Le crayon est initialement posé verticalement sur une table (en équilibre instable). On étudie sa chute à partir de cette position. Le coefficient de frottement statique entre le crayon et la table est supposé égale au coefficient de frottement dynamique et vaut \SI{0.7}{}.

\question Dans la première partie du mouvement, la pointe du crayon est immobile, en contact avec la table.
\subquestion Mettre le système en équation.
\subquestion Jusqu'à quelle date la pointe du crayon est-elle immobile ? Exprimer cette date sous la forme d'une intégrale.

\question Dans une seconde partie du mouvement, la pointe du crayon glisse avec frottement sur la table. Mettre le système en équation pour cette partie du mouvement.

\question Dans une dernière partie, la pointe du crayon se soulève et n'est plus en contact avec la table. Mettre le système en équation pour cette partie du mouvement.

