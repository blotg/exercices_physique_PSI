%%difficulte:   1
%%omis:         False
%%titre:        Montre
%%classes:      [PCSI]

\exercice{Montre}

Le fonctionnement d'une montre mécanique repose sur l'oscillation d'un pendule de torsion qui sert de base au calcul (mécanique) des secondes, minutes, heures, \dots. Celui-ci est composé d'un ressort en spirale et d'un volant d'inertie lui-même composé de deux masses de \SI{10}{g} situées toutes deux à \SI{1}{cm} de l'axe de rotation.

\question Calculer le moment d'inertie de du volant d'inertie par rapport à l'axe de rotation.

\question Quelle raideur doit avoir le ressort pour que la période des battements soit de \SI{1/3}{s}.

\question \subquestion Quelle incertitude doit avoir le moment d'inertie du volant d'inertie pour que la précision de la montre soit raisonnable ?
\subquestion Pour régler précisément sa valeur, les deux masses sont placées sur une tige filetée de pas \SI{0.1}{mm}. Quelle incertitude cela représente-t-il sur le nombre de tour sur cette vis ? Convertir en degrés et commenter.
