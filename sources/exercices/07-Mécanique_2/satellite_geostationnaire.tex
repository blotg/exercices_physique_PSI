%%difficulte:   2
%%omis:         False
%%titre:        Montre
%%classes:      [PCSI]

\exercice{Satellite géostationnaire}

Un satellite géostationnaire est un satellite qui est immobile dans le référentiel terrestre. Les satellites géostationnaires sont très utilisés dans le domaine des télécommunication.

\question
\subquestion Démontrer qu'un satellite géostationnaire doit nécessairement se trouver dans le plan de l'équateur.
\subquestion Démontrer qu'un satellite géostationnaire doit nécessairement être sur une orbite circulaire.

\question En raisonnant sur sa vitesse, quelle altitude doit nécessairement avoir le satellite géostationnaire ?

\question Déterminer l'énergie mécanique d'un satellite de masse $m$ sur une orbite géostationnaire.

On s'intéresse maintenant à la mise en orbite d'un tel satellite.

\question Déterminer l'expression de la première vitesse cosmique et donner sa signification.

\question Pourquoi préfère-t-on envoyer les satellites depuis un site proche de l'équateur ?

\question Dans quel sens de parcours de l'orbite géostationnaire est-il plus facile de placer un satellite ?

