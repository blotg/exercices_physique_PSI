%%difficulte:   3
%%omis:         False
%%titre:        Modulation de la largeur d'impulsion
%%classes:      [PCSI]

\exercice{Modulation de la largeur d'impulsion}

Pour commander la puissance d'une lampe halogène ou d'un moteur à courant continu, on peut les alimenter avec une tension périodique en créneau, dont la largeur est modulable. Ces deux dipôles (la lampe et le moteur) peuvent tous-deux être modélisés par une bobine en série avec une résistance.

\question Lorsque le signal créneau est à \SI{0}{V}, déterminer et résoudre l'équation différentielle qui régit l'évolution du système.

\question Même question lorsque le créneau est à sa valeur haute.

\question Tracer le courant en fonction du temps, dans deux cas : forte inductance et faible inductance.

\question Déterminer la valeur moyenne du courant.

\question Déterminer la puissance moyenne consommée par la lampe ou le moteur.
