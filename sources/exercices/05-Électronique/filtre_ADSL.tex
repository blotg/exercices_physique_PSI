%%difficulte:   2
%%omis:         False
%%titre:        Filtre ADSL
%%classes:      [PCSI]

\exercice{Filtre ADSL}

La technologie ADSL consiste à faire coexister sur une seule paire de fils électriques le signal audio correspondant au téléphone et le signal correspondant aux données internet s’étendant sur des fréquences supérieures à \SI{30}{kHz}.

Pour retrouver le signal audio à transmettre au haut-parleur du téléphone, on utilise un filtre dit "filtre ADSL". Ce filtre est constitué de la mise en cascade de deux filtres.

\question Dresser le gabarit du filtre ADSL.

\question Le premier filtre est un filtre LC, avec sortie sur le condensateur.
\subquestion Déterminer la fonction de transfert de ce filtre.
\subquestion Dresser son diagramme de Bode.

\question Le second filtre est un filtre (RL parallèle)-C, avec sortie sur le condensateur.
\subquestion Déterminer la fonction de transfert de ce filtre.
\subquestion Dresser son diagramme de Bode.

\question Sous quelle condition peut-on mettre les deux filtres en série sans que l'analyse précédente ne devienne invalide ?

\question Proposer des valeurs raisonnables des différents composants permettant de réaliser le filtre ADSL.
