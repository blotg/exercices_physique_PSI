%%difficulte:   4
%%omis:         False
%%titre:        Pertes sur les lignes électriques
%%classes:      [PCSI]

\exercice{Pertes sur les lignes électriques}

Une usine produisant de l’acier est reliée par une ligne électrique spécifique à un barrage hydro-électrique produisant l'énergie électrique dont elle a besoin.

On modélise le générateur du barrage par un générateur de tension idéal. La ligne électrique a une résistance de \SI{1}{\Omega}. L'usine a besoin de recevoir \SI{10}{kW}.

\question Proposer un schéma électrique modélisant la situation.

\question Si la tension reçue par l'usine est \SI{230}{V}, que vaut le courant électrique dans le câble ?

\question Même question pour \SI{10}{kV}.

\question Expliquer pourquoi on choisit de transporter la puissance électrique en utilisant une très haute tension.
