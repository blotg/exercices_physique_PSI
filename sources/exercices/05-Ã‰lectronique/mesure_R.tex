%%difficulte:   3
%%omis:         False
%%titre:        Mesure de résistance
%%classes:      [PCSI]

\exercice{Mesure de résistance}

On peut mesurer la valeur d'une résistance en l'alimentant et en mesurant la valeur de la tension à ses bornes et la valeur du courant qui la parcourt.

\question Proposer le schéma électrique d'un montage possible.

\question Présenter ce que sont un voltmètre et un ampèremètre idéaux. À quoi sont-ils équivalents ?

\question Proposer un modèle d'un voltmètre non idéal et d'un ampèremètre non idéal. Donner des ordres de grandeur des grandeurs introduites.

\question On branche le voltmètre "au plus près" de la résistance.
\subquestion En considérant le voltmètre et l'ampèremètre non-idéaux, déterminer la valeur mesurée de la résistance, obtenue en faisant le quotient de la tension mesurée par le voltmètre par le courant mesuré par l'ampèremètre.
\subquestion Quelle plage de résistance peut-on mesurer en ayant une erreur relative inférieure à \SI{1}{\%} ?

\question On branche l'ampèremètre "au plus près" de la résistance.
\subquestion En considérant le voltmètre et l'ampèremètre non-idéaux, déterminer la valeur mesurée de la résistance, obtenue en faisant le quotient de la tension mesurée par le voltmètre par le courant mesuré par l'ampèremètre.
\subquestion Quelle plage de résistance peut-on mesurer en ayant une erreur relative inférieure à \SI{1}{\%} ?
