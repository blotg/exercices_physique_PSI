%%difficulte:   2
%%omis:         False
%%titre:        Sélecteur de masse
%%classes:      [PCSI]

\exercice{Sélecteur de masse}

Un sélecteur de masse est un appareil dans laquelle on place un échantillon composé d'ions dont la fonction est de trier les ions de l'échantillon en fonction de leur masse.

Pour ce faire, l'appareil accélère les ions grâce à un champ électrique, puis les dévie grâce à un champ magnétique puis les fait passer dans un calibre afin de ne sélectionner que ceux qui ont la bonne masse.

\question Pour accélérer des cations, dans quel sens faut-il appliquer le champ électrique ?

\question On suppose que tous les cations présents ont une charge $+I$. On applique une tension de \SI{10}{kV} sur une distance de \SI{5}{cm} pour accélérer les particules. Exprimer la vitesse de sortie en fonction du nombre de la masse du cation.

\question Le cation accéléré arrive à présent dans un champ magnétique uniforme d'intensité \SI{700}{mT}. Exprimer la position du cation en fonction du temps.

\question Le champ magnétique est appliqué dans une zone de largeur \SI{3}{cm}. Déterminer l'angle avec lequel le cation sort de la machine.

\question Que dire de la précision et de la sélectivité de l'appareil ?
