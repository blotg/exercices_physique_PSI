%%difficulte:   2
%%omis:         False
%%titre:        Échasses
%%classes:      [PCSI]

\exercice{Échasses}

Un acrobate fait des échasses mais l'une de ses échasses se décroche. L'acrobate perd alors l'équilibre et commence à chuter. Le coefficient de frottement statique entre le bas d'une échasse et le sol est \SI{0.8}{}.

\question Mettre en équation le système.

\question Jusqu'à quel angle (exprimé sous forme d'une intégrale qu'on ne cherchera pas à calculer) le bas de l'échasse adhère-t-elle au sol ?

À partir de l'instant où l'échasse glisse, on peut considérer l'acrobate en chute libre.

\question Avec quelle vitesse entrera-t-il en contact avec le sol ?

\question Une fois remis de son incident, l'acrobate décide d'intégrer cette cabriole à sont numéro, mais pour ne pas se faire mal, il place un trampoline à l'endroit de sa chute.
\subquestion Où doit-il placer le trampoline ?
\subquestion Par quel système physique peut-on modéliser le trampoline ?
\subquestion Exprimer le recul du trampoline (son enfoncement maximal) en fonction de sa raideur et de la vitesse de l'acrobate.
\subquestion Quelle doit être la raideur du trampoline ?
\subquestion Quelle accélération subit alors l'acrobate ?
