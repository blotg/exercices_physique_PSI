%%difficulte:   2
%%omis:         False
%%titre:        Montagne russe
%%classes:      [PCSI]

\exercice{Montagne russe}

Une montagne russe est constituée d'une rampe inclinée à \SI{30}{\deg} qui hisse le train de \SI{1}{t} à une hauteur de \SI{30}{m}. Le circuit se compose ensuite d'un pente descendante (qui est abordée par le train à vitesse nulle) à \SI{60}{\deg} aboutissant en bas à un looping de \SI{10}{m} de haut. Le train est ensuite reconduit au point de départ par des virages, puis il est freiné par frottement solide.

\question On suppose ici que les frottements s'exerçant sur le train sont négligeables.
\subquestion Quelle vitesse a le train après avoir descendu la pente ?
\subquestion Même question en haut du looping.
\subquestion Quelle est l'accélération maximale subie par les passagers du train ?
\subquestion Quel rayon doivent avoir les virages pour que l'accélération qu'ils provoquent soit raisonnable ?

\question On suppose ici qu'il existe des frottements solides entre le train et les rails.
\subquestion Quelle valeur maximale peut avoir le coefficient de frottement dynamique pour que le train puisse arriver en haut du looping ?
\subquestion Que doit valoir le coefficient de frottement des freins pour arrêter le train sans provoquer une accélération insuportable ?
\subquestion Quelle distance met alors le train à s'arrêter ?

