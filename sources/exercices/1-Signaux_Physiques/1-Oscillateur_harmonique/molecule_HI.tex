%%theme:        Signaux Physiques
%%soustheme:    Oscillateur harmonique
%%difficulte:   3
%%omis:         False
%%titre:        Iodure d'hydrogène
%%classes:      [PCSI]

\exercice{Iodure d'hydrogène}

L'iodure d'hydrogène est une molécule diatomique constituée d'un atome d'hydrogène et d'un atome d'iode reliés par une liaison chimique modélisée par un ressort de raideur $k$. L'objet de l'exercice est de déterminer une valeur pour $k$.

Le spectre de l'iodure d'hydrogène présente une raie d'absorption à \SI{4.33}{\mu m}.

\question Déterminer la fréquence de vibration de la molécule en fonction de $k$. En déduire la valeur de $k$.
\question On suppose maintenant que la liaison est soumise à des frottements fluides qui représentent la perte d'énergie par rayonnement.
\subquestion Établir la nouvelle équation différentielle.
\subquestion Tracer l'allure de la solution dans chacun des cas possibles.
