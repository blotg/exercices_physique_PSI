%%theme:        Signaux Physiques
%%soustheme:    Optique géométrique
%%difficulte:   2
%%omis:         False
%%titre:        Fibre optique
%%classes:      [PCSI]

\exercice{Fibre optique}

Une fibre optique à sault d'indice est constituée d'une âme de diamètre \SI{1}{mm} et d'indice optique \SI{1.456}{} entourée d'une gaine de diamètre \SI{1.5}{mm} d'indice \SI{1.410}{}.

\question Quel peut être l'angle d'incidence d'une rayon lumineux dans la fibre optique pour qu'il y reste confiné ?

\question Comment s'appelle le cadre théorique utilisé ? Quelles sont ses limites ?

\question On désire faire l'image d'une source lumineuse sur l'entrée de la dibre optique avec une lentille convergente de vergence \SI{20}{\delta} placée à \SI{50}{cm} de l'entrée de la fibre.
\subquestion À quelle distance de l'entrée de la fibre doit être placée la source lumineuse ?
\subquestion Quelle diamètre peut avoir, au maximum la source lumineuse pour sa lumière rentre dans la fibre ?
\subquestion Que dire de la condition vue à la première question ?
\subquestion Sans changer la lentille, peut-on miniaturiser le système composé de la source, la lentille et l'entrée de la fibre à volonté ?
