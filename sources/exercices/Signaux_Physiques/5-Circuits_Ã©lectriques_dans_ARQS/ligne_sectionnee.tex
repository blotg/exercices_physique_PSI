%%theme:        Signaux Physiques
%%soustheme:    Électronique
%%difficulte:   3
%%omis:         False
%%titre:        Ligne sectionnée
%%classes:      [PCSI]

\exercice{Ligne sectionnée}

Il est fréquent que des évènements d'origine climatiques ou animale détériore des lignes téléphoniques, coupant ainsi certains habitant de connexion internet et téléphonique. Plutôt que d'inspecter la ligne minutieusement pour trouver l'endroit où se situe le défaut, les techniciens préfèrent estimer la longueur du câble les séparant de la coupure.

Pour ce faire, ils mesurent la capacité du condensateur formé par les deux fils en regard.

On peut mesurer la valeur d'une capacité de différentes manières.

\question On peut utiliser un dispositif chargeant le condensateur à courant constant et mesurer le temps que la tension à ses bornes met à atteindre une tension seuil.
\subquestion Exprimer la capacité du condensateur en fonction de la tension seul, du courant et de la durée.
\subquestion Si le câble possède, en plus de sa capacité, une inductance en série, cela modifie-t-il la mesure ?
\subquestion Si le câble possède, en plus de sa capacité, une résistance en série, cela modifie-t-il la mesure ?

\question On peut mettre à ses bornes une tension sinusoïdale et mesurer le courant qui le traverse.
\subquestion Exprimer la capacité du condensateur en fonction de la tension, du courant et de la fréquence.
\subquestion Si le câble possède, en plus de sa capacité, une inductance en série, cela modifie-t-il la mesure ?
\subquestion Si le câble possède, en plus de sa capacité, une résistance en série, cela modifie-t-il la mesure ? Comment peut-on alors corriger la mesure pour s'affranchir de cet effet ?

\question On peut utiliser ce condensateur dans un circuit RLC série, de résistance et d'impédance connues et mesurer sa fréquence de résonance.
\subquestion Exprimer la capacité du condensateur en fonction des grandeurs précédemment citées.
\subquestion Si le câble possède, en plus de sa capacité, une inductance en série, cela modifie-t-il la mesure ?
\subquestion Si le câble possède, en plus de sa capacité, une résistance en série, cela modifie-t-il la mesure ?
