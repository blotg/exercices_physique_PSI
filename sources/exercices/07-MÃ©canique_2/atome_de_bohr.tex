%%difficulte:   1
%%omis:         False
%%titre:        Montre
%%classes:      [PCSI]

\exercice{Atome de Bohr}

Dans le modèle de Bohr, l'atome d'hydrogène est constitué d'un électron en orbite circulaire autour d'un proton, soumis à la seule force d'attraction électrostatique exercée par le proton.

\question Exprimer le moment cinétique de l'électron en fonction de sa charge, de sa masse et du rayon de son orbite.

\question Exprimer l'énergie mécanique de l'électron en fonction de sa charge et du rayon de son orbite.

Bohr postule que le moment cinétique de l'électron autour du noyau est quantifié selon $L=,\hbar$ où $n\in \mathbb{N}$

\question Déduire l'expression de l'énergie mécanique en fonction de $n$.

\question Déterminer l'expression du rayon en fonction de $n$.

