%%difficulte:   3
%%omis:         False
%%titre:        Derby de démolition
%%classes:      [PCSI]

\exercice{Derby de démolition}

Aux États Unis, un Derby de démolition est une course automobile où les voitures s'entrechoquent en vue de s'infliger des dégâts et ainsi détruire celles des concurrents. La gagnante est la dernière en état de rouler.

Trois voitures (notées $A$, $B$ et $C$) restent dans le course. Chacune est à \SI{100}{m} des deux autres. $A$ fonce sur $B$, $B$ sur $C$ et $C$ sur $A$. Les voitures ont toutes une vitesse de \SI{50}{km/h}. Le coefficient de frottement statique entre les pneus et la piste est \SI{0.6}{}.

\question Dans un repère que vous choisirez, exprimer le vecteur vitesse de chaque voiture.

\question Déterminer la trajectoire de chaque voiture.

\question \subquestion Le mouvement d'une des voiture est-il une translation ?
\subquestion Le mouvement d'une des voiture est-il une rotation ?

\question Au bout de combien de temps a lieu le choc ?

\question Les pilotes parviennent-ils à poursuivre leur trajectoire jusqu'à l'impact ou leur voitures dérapent-elles avant ?

Quelques secondes avant l'impact, une des voitures pile, ses roues se bloquent. Le coefficient de frottement dynamique entre les pneus et la piste est de \SI{0.5}{}.

\question Quelle distance lui faut-il pour s'arrêter ? Combien de temps ?

